\documentclass[]{elsarticle} %review=doublespace preprint=single 5p=2 column
%%% Begin My package additions %%%%%%%%%%%%%%%%%%%
\usepackage[hyphens]{url}

  \journal{JEB? RSPB?} % Sets Journal name


\usepackage{lineno} % add
\providecommand{\tightlist}{%
  \setlength{\itemsep}{0pt}\setlength{\parskip}{0pt}}

\bibliographystyle{elsarticle-harv}
\biboptions{sort&compress} % For natbib
\usepackage{graphicx}
\usepackage{booktabs} % book-quality tables
%%%%%%%%%%%%%%%% end my additions to header

\usepackage[T1]{fontenc}
\usepackage{lmodern}
\usepackage{amssymb,amsmath}
\usepackage{ifxetex,ifluatex}
\usepackage{fixltx2e} % provides \textsubscript
% use upquote if available, for straight quotes in verbatim environments
\IfFileExists{upquote.sty}{\usepackage{upquote}}{}
\ifnum 0\ifxetex 1\fi\ifluatex 1\fi=0 % if pdftex
  \usepackage[utf8]{inputenc}
\else % if luatex or xelatex
  \usepackage{fontspec}
  \ifxetex
    \usepackage{xltxtra,xunicode}
  \fi
  \defaultfontfeatures{Mapping=tex-text,Scale=MatchLowercase}
  \newcommand{\euro}{€}
\fi
% use microtype if available
\IfFileExists{microtype.sty}{\usepackage{microtype}}{}
\ifxetex
  \usepackage[setpagesize=false, % page size defined by xetex
              unicode=false, % unicode breaks when used with xetex
              xetex]{hyperref}
\else
  \usepackage[unicode=true]{hyperref}
\fi
\hypersetup{breaklinks=true,
            bookmarks=true,
            pdfauthor={},
            pdftitle={Energetic Consequences of Sonar Exposure for Cetaceans},
            colorlinks=false,
            urlcolor=blue,
            linkcolor=magenta,
            pdfborder={0 0 0}}
\urlstyle{same}  % don't use monospace font for urls

\setcounter{secnumdepth}{0}
% Pandoc toggle for numbering sections (defaults to be off)
\setcounter{secnumdepth}{0}
% Pandoc header



\begin{document}
\begin{frontmatter}

  \title{Energetic Consequences of Sonar Exposure for Cetaceans}
    \author[1]{Max F Czapanskiy\corref{c1}}
   \ead{maxczap@stanford.edu} 
   \cortext[c1]{Corresponding author}
    \author[1]{Matthew S Savoca}
  
  
    \author[1]{Will T Gough}
  
  
    \author[1]{Jeremy A Goldbogen}
  
  
      \address[1]{Department of Biology, Hopkins Marine Station, Stanford University, 120
Ocean View Boulevard, Pacific Grove, CA 93950, USA}
  
  \begin{abstract}
  The sub-lethal effects of sonar exposure for cetaceans is not well
  enough understood to predict population consequences. To better inform
  conservation planning, we developed a model to predict the energetic
  costs of sonar avoidance behavior. When avoiding sonar sources,
  cetaceans may cease foraging and/or flee the area. We estimate the
  potential energy intake lost to foraging cessation and the additional
  locomotor costs from increased swim speeds.
  
  Although medium-sized cetaceans, especially beaked whales, make up the
  majority of mass strandings associated with naval sonar, our results
  indicate sub-lethal effects cause the most harm to large baleen whales.
  A severe response by a Cuvier's beaked whale (6.5 hour foraging
  cessation and 30 minutes of elevated swim speeds at 4.5 m/s) cost 49\%
  of its daily basal metabolic requirements. Conversely, a moderate
  response by a blue whale (1 hour foraging cessation and 5 minutes of
  elevated swim speeds at 2.5 m/s) cost more than nine times its daily
  basal metabolic requirements. These results may be used to reduce harm
  to endangered cetacean species.
  \end{abstract}
  
 \end{frontmatter}

\section{Introduction}\label{introduction}

Naval exercises involving sonar have been linked to mass strandings of
cetaceans worldwide since at least the 1980's (England et al., 2001;
Frantzis, 1998; Jepson et al., 2003; Simmonds and Lopez-Jurado, 1991).
However, the population consequences of sub-lethal effects are not as
well understood. Controlled exposure experiments (CEEs) show that
behavioral responses may include cessation of feeding and/or fleeing the
sonar source at elevated speed (DeRuiter et al., 2017, 2013;
Friedlaender et al., 2016; Goldbogen et al., 2013; Kvadsheim et al.,
2017; Southall et al., 2019; Tyack et al., 2011; Wensveen Paul J. et
al., 2019). Quantitatively linking these behaviors to demographics
requires an understanding of the impacts on individuals' health (Pirotta
et al., 2018). The mechanism addressed here is reduced energy stores due
to lost foraging opportunities and increased locomotor costs.

The two extant clades of cetaceans differ in their feeding styles.
Toothed whales (odontocetes) are raptorial feeders and locate prey using
echolocation {[}ref?{]}. Baleen whales (mysticetes) are filter feeders
and capture prey by lunging (Goldbogen et al., 2012) or continuous ram
filter feeding (Werth and Potvin, 2016). These feeding styles have
profound effects on feeding rate, energy per feeding event, dive depth,
and body size. Odontocetes feed at higher rates on smaller prey and most
larger odontocetes must dive to extreme depths to find sufficient prey
{[}ref?{]}. Lunge-feeding mysticetes (rorquals) engulf enormous
quantities of prey-laden water, increasing the energy intake per feeding
event but limiting dive depth and duration (Goldbogen et al., 2012).

\emph{feeding rates can be estimated from tagging (lunges, buzzes), prey
energy from active acoustics (rorquals) or the size of squid beaks and
fish otoliths (?) in stomach contents (odontocetes)}

Empirical estimates of cetacean metabolic rates are logistically
challenging for smaller species and infeasible for larger species.
Oxygen consumption has been measured for captive odontocetes trained to
swim under a metabolic hood using open-flow respirometry, showing that
mass-specific stroke costs are largely size invariant (Williams et al.,
1993, 2017). Whether these metabolic estimates scale to larger
odontocetes or mysticetes is unknown, so other methods of estimating
energy expediture include breathing rates and hydrodynamic models
(Potvin et al., 2012; Sumich, 1983).

Animals swim efficiently by cruising at 1-2 m/s and maintaining a
Strouhal number of 0.25 - 0.3 (Rohr and Fish, 2004; Sato Katsufumi et
al., 2007). The Strouhal number is a dimensionless quantity
\(St = \frac{Af}{U}\) where \(A\) is stroke amplitude, \(f\) is stroke
frequency, and \(U\) is swimming speed. Cetacean stroke amplitudes are
approximately one fifth body length {[}ref?{]} so there is a linear
relationship between swimming speed and stroke frequency for animals of
a given body size {[}gough?{]}.

\section*{References}\label{references}
\addcontentsline{toc}{section}{References}

\hypertarget{refs}{}
\hypertarget{ref-deruiter_multivariate_2017}{}
DeRuiter, S.L., Langrock, R., Skirbutas, T., Goldbogen, J.A.,
Calambokidis, J., Friedlaender, A.S., Southall, B.L., 2017. A
multivariate mixed hidden markov model for blue whale behaviour and
responses to sound exposure. The Annals of Applied Statistics 11,
362--392. \url{https://doi.org/10.1214/16-AOAS1008}

\hypertarget{ref-deruiter_first_2013}{}
DeRuiter, S.L., Southall, B.L., Calambokidis, J., Zimmer, W.M.X.,
Sadykova, D., Falcone, E.A., Friedlaender, A.S., Joseph, J.E., Moretti,
D., Schorr, G.S., Thomas, L., Tyack, P.L., 2013. First direct
measurements of behavioural responses by cuvier's beaked whales to
mid-frequency active sonar. Biology Letters 9, 20130223--20130223.
\url{https://doi.org/10.1098/rsbl.2013.0223}

\hypertarget{ref-england_joint_2001}{}
England, G.R., Evans, D., Lautenbacher, C., Morrissey, S., Hogarth, W.,
others, 2001. Joint interim report bahamas marine mammal stranding event
of 15-16 march 2000. US Department of Commerce, US Secretary of the
Navy.

\hypertarget{ref-frantzis_does_1998}{}
Frantzis, A., 1998. Does acoustic testing strand whales? Nature 392,
29--29. \url{https://doi.org/10.1038/32068}

\hypertarget{ref-friedlaender_prey-mediated_2016}{}
Friedlaender, A.S., Hazen, E.L., Goldbogen, J.A., Stimpert, A.K.,
Calambokidis, J., Southall, B.L., 2016. Prey-mediated behavioral
responses of feeding blue whales in controlled sound exposure
experiments. Ecological Applications 26, 1075--1085.
\url{https://doi.org/10.1002/15-0783}

\hypertarget{ref-goldbogen_scaling_2012}{}
Goldbogen, J.A., Calambokidis, J., Croll, D.A., McKenna, M.F., Oleson,
E., Potvin, J., Pyenson, N.D., Schorr, G., Shadwick, R.E., Tershy, B.R.,
2012. Scaling of lunge-feeding performance in rorqual whales:
Mass-specific energy expenditure increases with body size and
progressively limits diving capacity. Functional Ecology 26, 216--226.
\url{https://doi.org/10.1111/j.1365-2435.2011.01905.x}

\hypertarget{ref-goldbogen_blue_2013}{}
Goldbogen, J.A., Southall, B.L., DeRuiter, S.L., Calambokidis, J.,
Friedlaender, A.S., Hazen, E.L., Falcone, E.A., Schorr, G.S., Douglas,
A., Moretti, D.J., Kyburg, C., McKenna, M.F., Tyack, P.L., 2013. Blue
whales respond to simulated mid-frequency military sonar. Proceedings of
the Royal Society B: Biological Sciences 280, 20130657--20130657.
\url{https://doi.org/10.1098/rspb.2013.0657}

\hypertarget{ref-jepson_gas-bubble_2003}{}
Jepson, P.D., Arbelo, M., Deaville, R., Patterson, I.A.P., Castro, P.,
Baker, J.R., Degollada, E., Ross, H.M., Herráez, P., Pocknell, A.M.,
Rodríguez, F., Howie, F.E., Espinosa, A., Reid, R.J., Jaber, J.R.,
Martin, V., Cunningham, A.A., Fernández, A., 2003. Gas-bubble lesions in
stranded cetaceans. Nature 425, 575--576.
\url{https://doi.org/10.1038/425575a}

\hypertarget{ref-kvadsheim_avoidance_2017}{}
Kvadsheim, P.H., DeRuiter, S., Sivle, L.D., Goldbogen, J.,
Roland-Hansen, R., Miller, P.J., Lam, F.-P.A., Calambokidis, J.,
Friedlaender, A., Visser, F., Tyack, P.L., Kleivane, L., Southall, B.,
2017. Avoidance responses of minke whales to 1--4 kHz naval sonar.
Marine Pollution Bulletin 121, 60--68.
\url{https://doi.org/10.1016/j.marpolbul.2017.05.037}

\hypertarget{ref-pirotta_understanding_2018}{}
Pirotta, E., Booth, C.G., Costa, D.P., Fleishman, E., Kraus, S.D.,
Lusseau, D., Moretti, D., New, L.F., Schick, R.S., Schwarz, L.K.,
Simmons, S.E., Thomas, L., Tyack, P.L., Weise, M.J., Wells, R.S.,
Harwood, J., 2018. Understanding the population consequences of
disturbance. Ecology and Evolution 8, 9934--9946.
\url{https://doi.org/10.1002/ece3.4458}

\hypertarget{ref-potvin_metabolic_2012}{}
Potvin, J., Goldbogen, J.A., Shadwick, R.E., 2012. Metabolic
expenditures of lunge feeding rorquals across scale: Implications for
the evolution of filter feeding and the limits to maximum body size.
PLOS ONE 7, e44854. \url{https://doi.org/10.1371/journal.pone.0044854}

\hypertarget{ref-rohr_strouhal_2004}{}
Rohr, J.J., Fish, F.E., 2004. Strouhal numbers and optimization of
swimming by odontocete cetaceans. Journal of Experimental Biology 207,
1633--1642. \url{https://doi.org/10.1242/jeb.00948}

\hypertarget{ref-sato_katsufumi_stroke_2007}{}
Sato Katsufumi, Watanuki Yutaka, Takahashi Akinori, Miller Patrick J.O,
Tanaka Hideji, Kawabe Ryo, Ponganis Paul J, Handrich Yves, Akamatsu
Tomonari, Watanabe Yuuki, Mitani Yoko, Costa Daniel P, Bost
Charles-André, Aoki Kagari, Amano Masao, Trathan Phil, Shapiro Ari,
Naito Yasuhiko, 2007. Stroke frequency, but not swimming speed, is
related to body size in free-ranging seabirds, pinnipeds and cetaceans.
Proceedings of the Royal Society B: Biological Sciences 274, 471--477.
\url{https://doi.org/10.1098/rspb.2006.0005}

\hypertarget{ref-simmonds_whales_1991}{}
Simmonds, M.P., Lopez-Jurado, L.F., 1991. Whales and the military.
Nature 351, 448. \url{https://doi.org/10.1038/351448a0}

\hypertarget{ref-southall_behavioral_2019}{}
Southall, B.L., DeRuiter, S.L., Friedlaender, A., Stimpert, A.K.,
Goldbogen, J.A., Hazen, E., Casey, C., Fregosi, S., Cade, D.E., Allen,
A.N., Harris, C.M., Schorr, G., Moretti, D., Guan, S., Calambokidis, J.,
2019. Behavioral responses of individual blue whales (
\emph{balaenoptera musculus} ) to mid-frequency military sonar. The
Journal of Experimental Biology 222, jeb190637.
\url{https://doi.org/10.1242/jeb.190637}

\hypertarget{ref-sumich_swimming_1983}{}
Sumich, J.L., 1983. Swimming velocities, breathing patterns, and
estimated costs of locomotion in migrating gray whales,
\emph{eschrichtius robustus}. Canadian Journal of Zoology 61, 647--652.
\url{https://doi.org/10.1139/z83-086}

\hypertarget{ref-tyack_beaked_2011}{}
Tyack, P.L., Zimmer, W.M.X., Moretti, D., Southall, B.L., Claridge,
D.E., Durban, J.W., Clark, C.W., D'Amico, A., DiMarzio, N., Jarvis, S.,
McCarthy, E., Morrissey, R., Ward, J., Boyd, I.L., 2011. Beaked whales
respond to simulated and actual navy sonar. PLoS ONE 6, e17009.
\url{https://doi.org/10.1371/journal.pone.0017009}

\hypertarget{ref-wensveen_paul_j._northern_2019}{}
Wensveen Paul J., Isojunno Saana, Hansen Rune R., von Benda-Beckmann
Alexander M., Kleivane Lars, van IJsselmuide Sander, Lam Frans-Peter A.,
Kvadsheim Petter H., DeRuiter Stacy L., Curé Charlotte, Narazaki Tomoko,
Tyack Peter L., Miller Patrick J. O., 2019. Northern bottlenose whales
in a pristine environment respond strongly to close and distant navy
sonar signals. Proceedings of the Royal Society B: Biological Sciences
286, 20182592. \url{https://doi.org/10.1098/rspb.2018.2592}

\hypertarget{ref-werth_baleen_2016}{}
Werth, A.J., Potvin, J., 2016. Baleen hydrodynamics and morphology of
cross-flow filtration in balaenid whale suspension feeding. PLOS ONE 11,
e0150106. \url{https://doi.org/10.1371/journal.pone.0150106}

\hypertarget{ref-williams_physiology_1993}{}
Williams, T.M., Friedl, W.A., Haun, J.E., 1993. The physiology of
bottlenose dolphins (tursiops truncatus): Heart rate, metabolic rate and
plasma lactate concentration during exercise. Journal of Experimental
Biology 179, 31--46.

\hypertarget{ref-williams_swimming_2017}{}
Williams, T.M., Kendall, T.L., Richter, B.P., Ribeiro-French, C.R.,
John, J.S., Odell, K.L., Losch, B.A., Feuerbach, D.A., Stamper, M.A.,
2017. Swimming and diving energetics in dolphins: A stroke-by-stroke
analysis for predicting the cost of flight responses in wild
odontocetes. The Journal of Experimental Biology 220, 1135--1145.
\url{https://doi.org/10.1242/jeb.154245}

\end{document}


